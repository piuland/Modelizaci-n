\documentclass[12pt,reqno]{amsart}
\usepackage{latexsym,amsmath,amsfonts,amscd,amssymb,color}
\usepackage{graphics}

\setlength{\oddsidemargin}{15pt} \setlength{\evensidemargin}{15pt}
\setlength{\textwidth}{420pt} \setlength{\textheight}{630pt}
\setlength{\topmargin}{0pt}

\setlength{\parskip}{.15cm} \setlength{\baselineskip}{.5cm}
%\renewcommand{\baselinestretch}{1.2}

%\usepackage[T1]{fontenc}
%\usepackage{avant}
%\usepackage[cp1250]{inputenc}
%\usepackage{amsmath,amssymb,amsthm}
%\usepackage{caption}
%\usepackage{geometry}
%\usepackage{float}
%
%\usepackage{makeidx}
%\usepackage[matrix,arrow,curve]{xy}
%\usepackage{boxedminipage}
%\usepackage{epic}
%\usepackage{longtable}
%\usepackage{ifthen}
%\usepackage{tabularx}
%\usepackage{parskip}
%\usepackage{verbatim}
%\usepackage{tikz-cd}

\newcommand{\inc}{\hookrightarrow}
\newcommand{\la}{\langle}
\newcommand{\ra}{\rangle}
\newcommand{\x}{\times}
\newcommand{\ox}{\otimes}
\newcommand{\too}{\longrightarrow}
\newcommand{\bd}{\partial}


\newcommand{\cA}{\mathcal{A}}   %Abreviaciones para escribir letras tipo mathcal (caligr\'aficas)
\newcommand{\cO}{\mathcal O}
\newcommand{\cC}{\mathcal C}
\newcommand{\cE}{\mathcal E}
\newcommand{\cL}{\mathcal L}
\newcommand{\cD}{\mathcal{D}}
\newcommand{\cH}{\mathcal{H}}
\newcommand{\cM}{\mathcal{M}}
\newcommand{\cK}{\mathcal{K}}
\newcommand{\cF}{\mathcal{F}}

\newcommand{\ZZ}{\mathbb{Z}}
\newcommand{\CC}{\mathbb{C}}
\newcommand{\DD}{\mathbb{D}}
\newcommand{\CP}{\mathbb{C}P}
\newcommand{\RR}{\mathbb{R}}
\newcommand{\TT}{\mathbb{T}}
\newcommand{\QQ}{\mathbb{Q}}
\newcommand{\PP}{\mathbb P}   

\newcommand{\id}{\operatorname{id}}
\newcommand{\im}{\operatorname{im}}
\newcommand{\SU}{\operatorname{SU}}
\newcommand{\U}{\operatorname{U}}
\newcommand{\SO}{\operatorname{SO}}
\newcommand{\OO}{\operatorname{O}}   
\newcommand{\Sp}{\operatorname{Sp}} 
\newcommand{\GL}{\operatorname{GL}}
\DeclareMathOperator{\Hom}{Hom}
\DeclareMathOperator{\codim}{codim}
\DeclareMathOperator{\lcm}{lcm}
\DeclareMathOperator{\coker}{coker}
\DeclareMathOperator{\Id}{Id}
\DeclareMathOperator{\Diff}{Diff}
\DeclareMathOperator{\Sym}{Sym}
\DeclareMathOperator{\Ant}{Ant}
\DeclareMathOperator{\End}{End}

\renewcommand{\a}{\alpha}   %Abreviaciones de letras griegas para escribir m\'as r\'apido
\renewcommand{\b}{\beta}
\renewcommand{\d}{\delta}
\newcommand{\g}{\gamma}
\newcommand{\e}{\varepsilon}
\newcommand{\eps}{\epsilon}
\newcommand{\f}{\varphi}
\newcommand{\h}{\theta}
\renewcommand{\l}{\lambda}
\renewcommand{\k}{\kappa}
\newcommand{\s}{\sigma}
\newcommand{\m}{\mu}
\newcommand{\n}{\nu}
\renewcommand{\o}{\omega}
\newcommand{\p}{\phi}
\newcommand{\q}{\psi}
\renewcommand{\t}{\tau}
\newcommand{\z}{\zeta}
\newcommand{\G}{\Gamma}
\renewcommand{\O}{\Omega}
\renewcommand{\S}{\Sigma}
\newcommand{\D}{\Delta}
\renewcommand{\L}{\Lambda}
\renewcommand{\P}{\Phi}
\newcommand{\Q}{\Psi}
\newcommand{\vp}{\varphi} 
\renewcommand{\s}{\sigma} 

\newcommand{\ii}{\mathrm{i}}
\newcommand{\rId}{\mathrm{Id}} 

\newcommand{\frj}{\mathfrak{j}} 




\newcommand{\set}[1]{\left\{#1\right\}}

\newtheorem{theorem}{Teorema}   %Redefino estos comandos para que los ponga en espa\~nol.
\newtheorem{proposition}[theorem]{Proposici\'on}
\newtheorem{lemma}[theorem]{Lema}
\newtheorem{definition}[theorem]{Definici\'on}
\newtheorem{example}[theorem]{Ejemplo}
\newtheorem{note}[theorem]{Observaci\'on}
\newtheorem{corollary}[theorem]{Corolario}
\newtheorem{remark}[theorem]{Observaci\'on}
\newtheorem{conjecture}[theorem]{Conjetura}
\newtheorem{question}[theorem]{Pregunta}


\begin{document}

\begin{center}
\Large{\textbf{T\'itulo del trabajo}}
\end{center}


{\color{blue}

\begin{itemize}
\item Ver si las acciones de grupos en $X^4$ un $4$-orbifold suben a su resoluci\'on simpl\'ectica.

\item Pensar c\'omo resolver $X^6$ tal que sus puntos de isotrop\'ia est\'an todos contenidos
en un n\'umero $\le 2$ de divisores singulares.

\item Caso de $X^{2n}$ c\'iclico.

\item Caso de $X=M/\G$, donde $M$ es una variedad, $\G$ es un grupo resoluble con cocientes c\'iclicos
es decir existe $\G_k \lhd \cdots \lhd \G_2 \lhd \G_1=\G$
con $\G_{i+1}/\G_i$ un grupo c\'iclico.

\item Entender la resoluci\'on de singularidades cociente c\'iclicas $\CC^n/\G$
con $\G \cong \ZZ_m$. Primer paso, paper de N-P (singularidades aisladas).

\item Mirar m\'as referencias de resoluci\'on de orbifolds.
\end{itemize}

\section{Lemas}

%aquí pongo mis observaciones, que no se compilan en el pdf

\begin{lemma}
Let $\Psi: \RR^{2n} \to \RR^{2n}$ be a radial map of the form
$\Psi(x)=f(|x|) x$, with $f:\RR \to [0,\infty)$ a smooth function.

{\color{blue} En Espa\~nol: Sea $\Psi: \RR^{2n} \to \RR^{2n}$ una aplicaci\'on radial del tipo
$\Psi(x)=f(|x|) x$, con $f:\RR \to [0,\infty)$ una función de clase infinito.}

Let $\o$ be a symplectic form on $\RR^{2n}$ such that
$\o(\cdot, \frj \cdot) \ge 0$ is $\frj$-positive.

Then, $(\Psi^*\o) (\cdot, \frj \cdot)$ is $\frj$-semipositive
if and only if $f(r)+f'(r) \ge 0$.
\end{lemma}
\begin{proof}
Let $x \in \RR^{2n}$ and $r=|x|$.
Note first that 
\begin{align*}
d_x \Psi(r \bd_r) & = \bd_t|_{t=0} \Psi((t+1)x)) \\
& = \bd_t|_{t=0} f((t+1)r) \cdot (t+1)x=(f'(r) + f(r))x= (f'(r) + f(r))r \bd_r \, .
\end{align*}
Also, $d_x \Psi (v)= f(r) v$ for $v$ a vector tangent to the sphere,
i.e. orthogonal to $\bd_r$. This is because $\Psi$ is the homothecy $y \mapsto f(r)y$ between the spheres of radius $r$ and radius $f(r) r$.

Now take $u \in \RR^{2n}$ a tangent vector and write
$$
u=ar \bd_r + b r\frj (\bd_r) + cv
$$
with $v$ orthogonal to the plane $\langle \bd_r, \frj (\bd_r) \rangle$ generated by $\bd_r , \frj (\bd_r)$.
Then we have
\begin{align*}
\frj (u)  & = ar  \frj \bd_r - br \bd_r + c \frj v \, ; \\
d_x \Psi(u) &  = a(f'(r) + f(r)) r \bd_r + b r f(r) \frj (\bd_r) + cf(r) v \, ; \\
d_x \Psi(\frj (u)) & = ar f(r) \frj (\bd_r) - b (f'(r)+f(r))r \bd_r + cf(r)\frj (v) \, ; \\
J(d_x \Psi(u)) & = a(f'(r) + f(r)) r \frj (\bd_r)-br f(r) \bd_r + cf(r) \frj (v) \, .
\end{align*}
Hence
\begin{align*}
(\Psi^* \o)_x(u,\frj (u)) & =\o(d_x \Psi(u), d_x \Psi (\frj (u))) \\
& = f(r)(f'(r)+f(r))r^2 (a^2+b^2)\o(\bd_r,\frj (\bd_r)) + c^2 f(r)^2 \o(v,\frj (v))
\end{align*}
where we have used that $v$ belongs to $\langle \bd_r, \frj (\bd_r) \rangle^{\o \perp}$, the symplectic orthogonal.
\end{proof}
}

\section{Introducci\'on}

En este trabajo vamos a hacer cosas

\section{Preliminares}

Aqu\'i vamos a fijar algunas notaciones etc.

\section{Minimizaci\'on de longitudes y geod\'esicas}

En esta secci\'on vamos a estudiar condiciones necesarias que debe cumplir una curva para ser la de m\'inima longitud de entre todas las curvas que unen dos puntos $p,q \in \RR^n$.

\subsection{ Funcionales de energ\'ia y longitud.}

Recordemos que una aplicación $\g:[a,b] \to \RR^n$ se dice que es \emph{de clase uno}, o $\cC^1$,
si es derivable con derivada continua en $(a,b)$ y existen los l\'imites laterales
$\lim_{t \to a^+} \g'(t)$, $\lim_{t \to b^-} \g'(t)$. En este caso denotamos
$\g \in \cC^1([a,b], \RR^n)$.

\medskip

Para definir la longitud de una curva $\g$ no es realmente necesario que sea $\cC^1$ en todo su dominio $[a,b]$, 
sino que basta con que $[a,b]$ se pueda subdividir en subintervalos donde sea $\cC^1$. En tal caso
se dice que $\g$ es $\cC^1$ a trozos.

\begin{definition}
Dados dos puntos $p, q \in \RR^n$ definimos el \emph{espacio de caminos de $p$ a $q$}
como el conjunto de todas las aplicaciones $\cC^1$ a trozos que unen ambos puntos, es decir
$$
\O_{p,q}=\{ \g: [a,b] \to \RR^n \text{ t.q. } \g(a)=p, \g(b)=q \}
$$
siendo $[a,b] \subset \RR$ un intervalo, y $\g$ una aplicaci\'on $\cC^1$ a trozos.
\end{definition}

\begin{definition}
Definimos el funcional de energ\'ia como
$\cE: \O_{p,q} \to \RR$ dado por 
$$
\cE(\g)= \frac{1}{2} \int_a^b |\g'(t)|^2 dt \, .
$$
Definimos el funcional de longitud como $\cL: \O_{p,q} \to \RR$ dado por 
$$
\cL(\g)= \int_a^b |\g'(t)| dt \, .
$$
\end{definition}


\begin{proposition}
Si una curva minimiza el funcional de energ\'ia $\cE$, entonces 

\end{proposition}


\end{document}